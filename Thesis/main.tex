\documentclass[a4paper, 12pt]{article}

\documentclass[a4paper, 12pt]{article}
\usepackage{graphicx} % Required for inserting images
\usepackage{setspace}
\usepackage{geometry}
\usepackage{float}
\usepackage[polish]{babel}
\usepackage[T1]{fontenc}

\usepackage{biblatex}
\addbibresource{sources.bib}

\geometry
{
     top=1.5cm,
     bottom=2cm,
     left=2cm,
     right=2cm,
}
\linespread{1.1}


\begin{document}

\vspace*{\fill}

\begin{center}
    STRONA TYTUŁOWA
\end{center}

\vspace*{\fill}

\newpage

\section{Wstęp}

\section{Parametry opisujące jakość powietrza}

Pomimo spadającej liczby śmierci spowodowanych niską jakością powietrza 
wewnętrznego, co sugeruje ogólną poprawę jakości powietrza, nie można 
stwierdzić wyeliminowania tego problemu. Na całym świecie aż 4\% wszystkich 
zgonów przypisuje się właśnie zanieczyszczeniom powietrza wewnętrznego. 
Szczególnie narażone są kraje biedne i mniej rozwinięte technologicznie \cite{owid}.

Pomijając tak drastyczne przypadki zła kondycja powietrza może powodować szereg mniej 
groźnych problemów zdrowotnych, takich jak pogorszenie zdolności rozwiązywania 
problemów i podejmowanie decyzji \cite{co2-effects}, co w szczególności dotyka placówki oświatowe i 
osoby tam przebywające. Bacząc na powyższe, oraz na fakt że człowiek średnio spędza większość dnia w swoim 
domu \cite{time-indoors}, nie licząc takich aktywności jak praca w zamkniętym pomieszczeniu, nowoczesne 
budynki powinny umożliwiać monitorowanie stanu powietrza wewnątrz 
ich w celu lepszej kontroli nad ich jakością.

Stan jakości powietrza wewnętrznego można opisać szeregiem parametrów: przepływ powietrza, 
pomiar stężenia cząsteczek stałych, tlenków węgla czy dwutlenku siarki. Jednymi z 
najczęściej stosowanych parametrów są jednak temperatura, wilgotność powietrza oraz 
zawartość dwutlenku węgla w powietrzu. 

\subsection{Temperatura}

Komfortowy przedział temperatur dla człowieka w zamkniętych pomieszczeniach w dużej 
mierze zależy od otaczającego klimatu i przyzwyczajeń z tego wynikających. Zależy on od takich czynników 
jak wiek, płeć, pora roku, klimat na danym obszarze czy rodzaj wykonywanej pracy.

W Polsce za minimalną dopuszczalną temperaturę w miejscu pracy przyjmuje się 18st. Celsjusza dla 
lekkiej pracy niewymagającej wysiłku fizycznego oraz 14st. dla pomieszczeń, w których wykonywane są prace 
fizyczne \cite{manutan-bhp}. Aktualne przepisy BHP nie normują jednak maksymalnej dopuszczalnej temperatury. 
Kolejnym dokumentem pomagającym przybliżyć komfortowy przedział temperatur w pomieszczeniach jest
"Rozporządzenie Min. Infrastruktury z 12.04.2002 w sprawie warunków technicznych jakim powinny odpowiadać 
budynki i ich usytuowanie", a dokładnie art. 134 punkt 2 \cite{rozp-bud}. Określono tam temperaturę na potrzeby obliczeń szczytowej 
mocy cieplnej w pomieszczeniach zależnie od ich przeznaczenia.
Dla przykładu, pomieszczenia przeznaczone na stały pobyt ludzi bez okryć zewnętrznych, nie wykonujących 
w sposób ciągły pracy fizycznej posiadają temperaturę obliczeniową 20st.C:

\begin{figure}[H]
    \caption{Fragment tabeli z Rozporządzenia przedstawiający temperatury obliczeniowe dla niektórych typów pomieszczeń}
    \includegraphics[width=\textwidth]{zdj/min-tabela.png}
\end{figure}

Za niska temperatura może powodować zwiększoną śmiertelność z powodu chorób układu oddechowego czy podwyższone 
ciśnienie \cite{who-cold}. Za wysoka zaś oprócz oczywistego dyskomfortu może doprowadzić zwiększonej śmiertelności, 
szczególnie u osób z chorobami układu oddechowego czy cierpiących na arytmię \cite{bmj-heat}.

W tej pracy za optymalny przedział temperatur przyjmuje się 18-22st. Celsjusza.

\subsection{Wilgotność}

Wilgotność powietrza można mierzyć na parę sposobów, natomiast odpowiednie aspekty tej pracy będą 
skupiały na wilgotności względnej - jest to stosunek wilgotności bezwzględnej powietrza do jej wartości 
maksymalnej w danej temperaturze \cite{termodynamika}

W pomieszczeniach biurowych o regulowanej temperaturze za optymalny zakres przyjmuje się 
od 25 do 60 \% wilgotności \cite{inz-bud}. Niższa wilgotność, szczególnie w połączeniu z niską temperaturą 
mogą przyczyniać się do zwiększonego występowania chorób układu oddechowego \cite{low-hum}. Za wysoka wilgotność 
zaś sprzyja rozwojowi i rozprzestrzenianiu się związków biologicznych (bakterie, wirusy, grzyby) 
powodujących alergie i choroby zakaźne \cite{high-hum}.

Podobnie jak temperatura, wilgotność zależy od wielu czynników i może zmieniać się bardzo dynamicznie, 
nawet w ciągu dnia. Wpływanie na wilgotność powietrza polega przede wszystkim na możliwości odczytu 
jego aktualnej lub historycznej wartości.

\begin{figure}[H]
    \includegraphics[width=\textwidth]{zdj/wilgotnosc_powietrza_w_pomieszczeniach.jpg}
    \caption{Wpływ wilgotności powietrza na nasilenie poszczególnych czynników}
\end{figure}

W związku z powyższym za optymalny przedział przyjmuje się od 30\% do 50\% wilgotności względnej 
powietrza.

\subsection{Dwutlenek węgla}

Podobnie jak w przypadku wilgotności, zawartość CO2 w powietrzu można mierzyć w wielu jednostkach. 
W tej pracy zdecydowano się jednak posługiwać się jednostką ppm - części na milion.

Za standard bezpiecznego poziomu CO2 w pomieszczeniu przyjmuje się wskaźnik Pettenkofera, który określa, 
że bezpieczne maksymalne stężenie tego związku w powietrzu wyrażone w częściach na milion wynosi 1000ppm \cite{pettenhofer}.

Rozporządzenie ministra rodziny, pracy i polityki społecznej z dnia 12 czerwca 2018 r. w sprawie 
najwyższych dopuszczalnych stężeń i natężeń czynników szkodliwych dla zdrowia w środowisku \cite{min-stezenia} pracy określa 
dopuszczalne stężenia szkodliwych dla człowieka substancji w trakcie pracy. Dla CO2 jest to:

* 9000 mg/m3 (~4,917ppm) - wartość średnia ważona, w ciągu 8-godzinnego dobowego i przeciętnego tygodniowo wymiaru pracy
* 27000 mg/m3 (~14,752ppm) - maksymalne stężenie, które może występować nie dłużej niż 15min, nie częściej niż 2 razy w ciągu zmiany i w odstępie dłuższym niż 1 godzina.

Standardowa ilość atmosferycznego dwutlenku węgla w powietrzu wynosi aktualnie około 417ppm. 
Liczba ta od początku Rewolucji Przemysłowej w 1750r. stale rośnie \cite{atmo-co2-change}, a według niektórych danych jest 
ono nawet 50\% wyższe niż sprzed wspomnianej daty \cite{50-percent}. Stężenie CO2 w powietrzu wyższe od 
wspomnianego wskaźnika Pettenkofera może powodować trudności w koncentracji, senność czy 
trudności z oddychaniem \cite{pettenhofer}.

\begin{figure}[H]
    \centering
    \includegraphics[width=0.8\textwidth]{zdj/more-co2.png}
    \caption{Wzrost stężenia dwutlenku węgla w atmosferze na przestrzeni lat}
\end{figure}

Za optymalny przedział wartości zawartości dwutlenku węgla w powietrzu przyjmuje się 400-1000ppm.

\section{Bezprzewodowy system monitoringu parametrów jakości powietrza}

\subsection{Technologia IQRF}

IQRF jest to bezprzewodowa technologia Mesh pracująca w pod-gigaherzowych pasmach ISM. Nie wymaga żadnej zewnętrznej infrastruktury, 
licencji i opłat dostawcy \cite{what-iqrf}. Zamontowanie transceivera IQRF w kompatybilnych urządzeniach pozwala im na komunikację między sobą
poprzez przesyłanie pakietów danych. 

Każda sieć wykorzystująca IQRFMESH potrzebuje koordynatora do odpowiedniego działania. Urządzenia w sieci komunikują się na zasadzie
master(koordynator) - slave(węzeł). Węzły nie mogą komunikować się ze sobą bezpośrednio, natomiast w topologi innej niż
Gwiazda (np. Mesh) węzły wspierają routowanie pakietów do innych węzłów \cite{iqrf-rules}.

IQRFMESH jest protokołem umożliwiającym, między innymi, zwiększenie zasięgu sieci poprzez umożliwienie węzłom routowania pakietów.
Jest możliwe połączenie urządzeń bez wykorzystania tego protokołu, ale na potrzeby tej pracy jest to zalecane \cite{iqrfmesh}. Urządzenia w 
sieci IQRFMESH komunikują się poprzez protokół DPA, w której wiadomości są przesyłane w strukturze bajtowej \cite{dpa-guide}.

Każda z wiadomości przesyłanych protokołem składa się co najmniej z:
\begin{itemize}
    \item NADR - 2-bajtowy adres urządzenia docelowego
    \item PNUM - 1-bajtowy adres urządzenia peryferialnego w urządzeniu docelowym. Peryferium może być na przykład czujnik temperatury bądź wbudowana dioda LED
    \item PCMD - 1-bajtowa komenda, której wykonanie zleca się urządzeniu. 
    \item HWPID - 1-bajtowy Identyfikator Profilu Urządzenia (HardWare Profile ID). Wartość stała określająca rodzaj/model urządzenia. Dla wykorzystywanych w projekcie
czujników jest to "4001".
\end{itemize}

oraz opcjonalnie z pola DATA, które może zawierać maksymalnie 56 bajtów danych. Cała wiadomość, łącznie z obowiązkowymi parametrami wymienionymi wyżej nie może przekroczyć
64 bajtów 

Przykład wiadomości:

\begin{center}
    \textbf{01005E010140FFFFFFFF}
\end{center}

jeżeli rozdzielimy ją na poszczególne parametry uzyskujemy:

\begin{table}[h]
    \centering
    \begin{tabular}{|c|c|c|c|c|}
    \hline
    NADR & PNUM & PCMD & HWPID & DATA \\ \hline
    0100 & 5E & 01 & 0140 & FFFFFFFF \\ \hline
    \end{tabular}
\end{table}

Protokół DPA stosuje ustawienie bajtów typu little-endian. W związku z tym bajty każdego parametru dłuższego
niż 2 bajty są pisane "od końca". Przykładowo, jeżeli chcemy zaadresować urządzenie o NADR równym "0001", to 
do ramki wiadomości odpowiadającej temu adresowi wpisuje się "0100". Taka sama sytuacja ma miejsce z parametrem
HWPID, który również ma długość większą niż 1 bajt.

Powyższa wiadomość pochodzi z dokumentacji czujnika Protronix używanego w projekcie \cite{protronix-comms}, a jej 
wysłanie powoduje zebranie z czujnika o adresie 0001 danych ze wszystkich dostępnych czujników - temperatury, 
wilgotności, dwutlenku węgla i baterii, jeżeli jej stan jest niski.

\subsection{Czujniki i bramka IQRF}

Urządzeniami bezpośrednio odpowiedzialnymi za pobieranie z otoczenia danych o jakości powietrza są cztery czujniki NLB-CO2+RH+T-5-IQRF firmy
Protronix. Każdy czujnik jest w stanie odczytać temperaturę, wilgotność względną oraz zawartość dwutlenku węgla w powietrzu. Zasilanie jest zapewniane
bateryjnie poprzez dwie baterie AA 1.5V, a komunikacja odbywa się poprzez zamontowany w gnieździe SIM nadajnikoodbiornik (transceiver) IQRF. Każdy taki sensor stanowi węzeł sieci IQRFMESH.

\begin{figure}[H]
    \centering
    \includegraphics[width=0.5\textwidth]{zdj/NLB-CO2RHT-5-IQRF.jpg}
    \caption{Podglądowe zdjęcie wykorzystywanego czujnika \cite{protronix-product}}
\end{figure}

Koordynatorem sieci jest bramka GW-ETH02 zasilana sieciowo napięciem 5VDC. Urządzenie łączy się z Internetem poprzez gniazdo Ethernet oraz
posiada gniazdo karty SD jako jedną z opcji zapisu danych. Bramka posiada wbudowany transceiver IQRF.

\begin{figure}[H]
    \centering
    \includegraphics[width=0.5\textwidth]{zdj/gateway.png}
    \caption{Podglądowe zdjęcie bramki \cite{gateway-product}}
\end{figure}

\subsubsection{Konfiguracja}

Do konfiguracji urządzeń wchodzących w skład sieci posłużono się narzędziem IQRF IDE w wersji 4.70. Narzędzie to umożliwia między 
innymi konfigurację modułów IQRF TR zawartych w urządzeniach oraz odczyt pewnych danych o transceiverach IQRF, takich jak wersja systemu
operacyjnego czy protokołu DPA.

Bramka może pracować w dwóch trybach: Datalogger lub Gateway. Konfiguracja bramki jako Datalogger powoduje przesłanie wszystkich danych,
wchodzących i wychodzących, do serwera IQRF Cloud. Z kolei konfiguracja bramki jako Gateway nie pozwala na korzystanie z IQRF Cloud, za to wymaga 
zdalnego połączenia przez protokół UDP z urządzeniem końcowym.

Jako, iż serwer IQRF Cloud zapewnia wygodę w postaci dobrze udokumentowanego API, możliwości interakcji z bramką z poziomu strony oraz swego
rodzaju buffer w postaci 2000 rekordów, w pracy zdecydowano się użyć ustawienia Datalogger.

Tego jak i innych potrzebnych ustawień można dokonać wybierając opcje TR Configuration w IQRF IDE podczas, gdy bramka jest fizycznie podłączona
do komputera. Parametry bramki zostały ustawione na wartości przewidziane przez producenta czujników Protronix \cite{protronix-comms}. Upewniono 
się również, że bramka będzie działała z urządzeniami LP (Low Power), poprzez wybranie na panelu odpowiedniego ustawienia sieci STD+LP.

Następnie każdy z transceiverów przeznaczonych do montażu w czujnikach został tymczasowo zamontowany w programatorze CK-USB-04A będącego częścią
zestawu IoT Starter Kit w celu sprawdzenia poprawności ustawień transceivera zgodnie z zaleceniami producenta.

\begin{figure}[H]
    \centering
    \includegraphics[width=0.7\textwidth]{zdj/protronix-settings.png}
    \caption{Fabryczna konfiguracja modułów TR. Wycinek z \cite{protronix-comms}}
\end{figure}

\subsubsection{Topologia sieci}

To, w jakiej topologii będą pracowały czujniki sieci IQRF MESH zależy od odległości między czujnikiem a bramką. W przypadku 
mniejszych dystansów wszystkie czujniki będą bezpośrednio połączone bezprzewodowo z bramką - topologia gwiazdy. Jeżeli natomiast
odległości są znaczne niektóre węzły sieci mogą sprawować również funkcję routerów, przekierowujących pakiety między innymi 
węzłami i bramką. W takim przypadku mamy do czynienie z topologią siatki (ang. mesh).

\begin{figure}[H]

\centering
\begin{subfigure}{0.4\textwidth}
    \centering
    \includegraphics[width=0.8\linewidth]{zdj/star-top.png}
    \caption{Topologia gwiazdy \cite{fig-star-top}}
\end{subfigure}
\begin{subfigure}{0.4\textwidth}
    \centering
    \includegraphics[width=0.8\linewidth]{zdj/mesh-top.png}
    \caption{Topologia siatki \cite{fig-mesh-top}}
\end{subfigure}
   
\caption{Schematy porównujące obie topologie sieci}

\end{figure}

[ZDJĘCIE Z IQRF IDE PANEL MESH]

\subsection{Serwer IQRF Cloud}

IQRF Cloud jest serwisem w chmurze umożliwiającym komunikację z bramką IQRF po jej zarejestrowaniu w tymże serwisie, w którym uprzednio należy
założyć konto. Po rejestracji serwis umożliwia obustronną komunikację z urządzeniem w postaci wiadomości protokołu DPA. Aby takową nadać należy 
wejść w panel zarejestrowanej bramki i wybrać opcję "Send command to gateway":

\begin{figure}[H]
    \centering
    \includegraphics[width=0.4\textwidth]{zdj/send-command-to-gw.png}
    \caption{Opcja "Send command to gateway"}
\end{figure}

Po wybraniu tej opcji otrzymujemy widok umożliwiający wysyłanie do brami wiadomości protokołem DPA, a każde nadanie wymaga dodatkowo podania
hasła do bramki wybieranego na etapie konfiguracji.

\begin{figure}[H]
    \includegraphics[width=\textwidth]{zdj/cloud-send-message.png}
    \caption{Fragment panelu umożliwiający wysłanie danych z serwera chmury do bramki}
\end{figure}

Dane odbierane/wysyłane przez serwer i bramkę są widoczne w panelu głównym odpowiednim dla bramki. Na serwerze może istnieć jednocześnie tylko 2000
rekordów. Po przekroczeniu tego limitu najstarsze dane są nadpisywane.

Każdy rekord skład się z:
\begin{itemize}
    \item indeksu 
    \item czasu otrzymania przez serwer i bramkę
    \item kierunku przesyłu danych (transmisja a odbiór)
    \item długości danych wyrażonej w bajtach 
    \item samych danych w formacie szesnastkowym (znaki 0-9 i A-F)
    \item statusu odebrania przez bramkę (Sent, Confirmed, Expired...)
    \item indeksu w bazie danych serwera.
\end{itemize}

\begin{figure}[H]
    \includegraphics[width=\textwidth]{zdj/cloud-records.png}
    \caption{Fragment tabeli otrzymanych wiadomości w IQRF Cloud}
\end{figure}

Powyższy panel jednak stanowi dla tej pracy bardziej narzędzie diagnostyczne niż kluczowy element działania systemu - z poziomu działania aplikacji
webowej nie da się wejść w interakcję z tym panelem, a przynajmniej nie jest to przewidziane. Serwis udostępnia więc własny interfejs 
programistyczny (API), który umożliwia przesyłanie zapytań z innego urządzenia do tego serwisu i w konsekwencji do bramki oraz czujników.

Każde zapytanie wysyłane do serwera składa się z adresu bazowego na jaki przesyłamy zapytania, wymaganych i opcjonalnych parametrów oraz tzw. 
sygnatury. Adresem bazowym na dzień pisania pracy jest \textbf{https://cloud.iqrf.org/api/api.php?}.

\subsubsection{Parametry wymagane i opcjonalne}

Lista parametrów wymaganych prezentuje się następująco:

\begin{itemize}
    \item ver - aktualna wersja API. W trakcie pisania pracy jest to "2" 
    \item uid - login użytkownika do serwisu IQRF Cloud
    \item gid - ID zarejestrowanej bramki, z którą chcemy się komunikować. Widoczne zarówno w portalu jak i na obudowie bramki
    \item gpw - hasło do bramki. Wymagane jedynie w przypadku stosowania komendy "add"
    \item cmd - jedna z komend: dnld, uplc, dnlc
    \item data - dane do przesłania do bramki
    \item signature - podpis potwierdzający użytkownika.
\end{itemize}

Parametry opcjonalne dają zaś kontrolę nad ilością pobieranych danych:

\begin{itemize}
    \item from - indeks pierwszego rekordu, który chcemy pobrać
    \item to - indeks ostatniego rekordu, który chcemy pobrać
    \item count - ilość pobieranych rekordów
    \item time\_from - data utworzenia pierwszego rekordu, który chcemy pobrać 
    \item time\_to - data utworzenia ostatniego rekordu, który chcemy pobrać
    \item new - pobieranie danych zaczynając od ostatniego pobranego rekordu
    \item last - pobieranie rekordów zaczynając od najpóźniejszych
\end{itemize}

Bardziej szczegółowy opis wszystkich parametrów oraz ograniczenia w ich stosowaniu dostępne są w dokumentacji IQRF Cloud Serwer Technical Guide 
\cite{iqrfcloud-guide}

\subsubsection{Sygnatura zapytania}

Jakakolwiek komunikacja z serwerem wymaga podaniu parametru \textbf{signature}, który gwarantuje bezpieczeństwo przesyłu danych przez API 
potwierdzając prawo dostępu użytkownika do danych. Dokumentacja \cite{iqrfcloud-guide} definiuje sygnaturę następująco:

\begin{figure}[H]
    \includegraphics[width=\textwidth]{zdj/md5.png}
\end{figure}

Gdzie: 

\begin{itemize}
    \item parameter\_part - ciąg znaków adresu URL zapytania pomiędzy adresem bazowym a sygnaturą. Na przykład, dla zapytania \\
"https://cloud.iqrf.org/api/api.php?ver=2\&uid=xxx\&gid=xxx\&cmd=dnld\&signature=xxx" \\ 
wartość ta jest równa "ver=2\&uid=xxx\&gid=xxx\&cmd=dnld"
    \item api\_key - klucz API dostępny po zalogowaniu się do portalu IQRF Cloud
    \item ip\_address - adres IP klienta
    \item timestamp - liczba sekund począwszy od 01.01.1970 00:00UTC podzielona przez 600. Podzielenie przez 600 zapewnia 10-minutową
ważność każdej sygnatury.
\end{itemize}

'md5' jest zaś kryptograficzną funkcją haszującą, której argumentem jest wyżej przedstawiony ciąg znaków 


\begin{figure}[H]
    \includegraphics[width=\textwidth]{zdj/api-key.png}
    \caption{Fragment widoku profilu użytkownika. Czerwoną strzałką wskazano klucz API.}
\end{figure}

Kompletne zapytanie składa się z adresu bazowego, ciągu parametrów oraz sygnatury. Na przykład:
\begin{center}
    \begin{small}
        https://cloud.iqrf.org/api/api.php?ver=2\&uid=12345678\&gid=ABCD1234\&cmd=dnld\&signature=xxx"
    \end{small}
\end{center}

jest ważnym zapytaniem API i spowoduje pobranie maksymalnie 500 rekordów z serwera, zakładając poprawność wszystkich parametrów.

\subsection{Aplikacja webowa}

Aplikacja webowa stanowi centralny obiekt tej pracy i jest ostatnim elementem wchodzącym w skład systemu monitorowania danych
z czujników. 

Serwer tej aplikacji jest odpowiedzialny za wysyłanie zapytań do serwisu IQRF Cloud w celu poboru danych z czujników, pobieranie
tych danych z serwisu, walidacje, filtrowanie, formatowanie i zapisywanie do bazy danych. Zapewnia również szereg ścieżek API pozwalających
klientowi na interakcje z zasobami zapisanymi w bazie danych (odczyty z czujników i ich metadane). Zastosowanie bazy danych 
jest konieczne ze względu na fakt, że serwis IQRF Cloud przechowuje tylko 2000 najnowszych rekordów.

Klient natomiast zapewnia czytelny interfejs graficzny w postaci szeregu widoków umożliwiających przeglądanie aktualnych i 
historycznych danych oraz ich pobieranie.

Dokładniejszy opis działania aplikacji webowej zostanie przedstawiony w dalszej części pracy.

\subsection{Opis działania całego systemu}

Typowy cykl działania systemu jest wywoływany co 15 minut, a prezentuje się on następująco:

\begin{enumerate}
    \item Serwer aplikacji wysyła zapytanie do IQRF Cloud o pobranie danych z czujników.
    \item IQRF cloud otrzymuje zapytanie o odczyt czujników i przekazuje je do bramki
    \item Czujniki otrzymują wiadomość przesłaną przez bramkę i odsyłają odczyt z czujników z powrotem do bramki
    \item Bramka przesyła odczyty do serwera IQRF Cloud gdzie są tymczasowo zapisywane.
    \item Serwer aplikacji webowej wysyła kolejne zapytanie, tym razem o przesłanie danych z IQRF cloud
do aplikacji.
    \item Dane pobrane na serwer aplikacji są kolejno poddawane walidacji, filtrowaniu i formatowaniu.
    \item Przygotowane dane są zapisywane w bazie danych 
    \item Klient pobiera nowe dane z bazy
    \item Dane prezentowane na widokach ulegają odświeżeniu pod wpływem zmiany dostępnych danych.
\end{enumerate}

\section{Bibliografia}

\printbibliography

\end{document} 