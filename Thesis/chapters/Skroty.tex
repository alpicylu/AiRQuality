\section{Opis używanych pojęć}

Opisy skrótów i terminów użytych w pracy. Wyjaśnienia niektórych z nich mogą zostać rozwinięte w dalszych
rozdziałach pracy:
\begin{itemize}
    \item Framework - struktura lub platforma programistyczna, która 
    zapewnia ogólny szkielet do tworzenia, rozwijania i utrzymania aplikacji internetowych
    \item API - Application Programming Interface, zestaw funkcji, które umożliwiają różnym 
    programom bądź ich komponentom komunikację między sobą. W kontekście aplikacji webowych, API zapewnia 
    interakcję pomiędzy frontendem a backendem oraz między różnymi serwisami internetowymi.
    \item Backend - jedna z dwóch głównych warstw aplikacji webowej. Backend to część, która 
    działa po stronie serwera, obsługując logikę biznesową, bazę danych, autentykację, autorizację i inne funkcje, 
    niewidoczne dla użytkownika końcowego
    \item Frontend - część, z którą użytkownik bezpośrednio interaguje, obejmująca interfejs użytkownika.
    Zajmuje się prezentacją danych oraz interakcją z użytkownikiem.
    \item serwis IQRF Cloud - zewnętrzny serwer zapewniony przez Complete Internet Services s.r.o. Spełnia on 
rolę pośrednika pomiędzy aplikacją a systemem czujników wraz z bramką umożliwiając zbieranie i zapis danych z czujników 
    
\end{itemize}