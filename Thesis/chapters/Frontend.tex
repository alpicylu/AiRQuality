\section{Widoki aplikacji i opis ich działania}

\subsection{Widok główny}
Widok główny pozwala na ogólną prezentację danych z czujników z domyślnego zakresu 24-godzin z zaznaczeniem, w której sali znajduje się dany czujnik. 
Użytkownik posiada opcję wyboru jaki typ odczytu będzie prezentowany na wykresie (temperatura, wilgotność względna, stężenie CO2), zarówno indywidualnie 
poprzez interakcję z guzikami "radiowymi" dla każdego czujnika/sali oraz dla każdego z czujników jednocześnie poprzez wybranie typu
w liście rozwijanej. 

[SCREEN Z WIDOKU]

Dane z czujników przedstawione na wykresach są aktualizowane tak często, jak stają się one dostępne, biorąc pod uwagę opóźnienia związane
z cyklicznym odpytywaniem serwera o nowe dane czy straty czasu związane z samym przesyłem danych. Stan jakości powietrza jest dodatkowo
odzwierciedlany poprzez zmienne tło na wykresach - zielony oznacza optymalną wartość parametru, pomarańczowy stan bliski wartościom powodującym
dyskomfort bądź negatywnie wpływającym na zdrowie, zaś czerwony wartości niezalecane przy długotrwałym przebywaniu w pomieszczeniu.

Przedziały zostały dobrane na podstawie rozdziału "\nameref{opis-parametrow}"

\subsection{Widok szczegółów}

Po kliknięciu na dowolny z elementów odpowiadających czujnikowi na poprzednim widoku ukazuje się nam widok szczegółów danego czujnika. Można tutaj odczytać
bieżące pomiary wszystkich trzech typów, podstawowe statystyki odczytów z danego zakresu (wartości: aktualna, maksymalna, minimalna, średnia arytmetyczna), 
podać zakres pomiarów do wyświetlenia oraz pobrać plik CSV zawierający dane z podanego zakresu.

[Widok]

Wybór zakresu odbywa się poprzez wybranie dwóch dat - początkowej i końcowej - w lewej dolnej części widoku. Daty można wpisać ręcznie lub posłużyć się
selektorem dat pojawiającym się po naciśnięciu w element. Aby wyświetlić dane z wybranego przedziału należy kliknąć w przycisk "Apply", po czym z serwera pobierane
są odpowiednie dane i ładowane do widoku. W tym trybie nowe dane nie są pobierane z serwera. 
Aby powrócić do domyślnej formy prezentowania odczytów (ładowanie danych na bieżąco) należy nacisnąć przycisk "Default". Bieżące oraz zaległe dane z odczytów
zaczną znowu być ładowane do widoku.

[Wybór daty]

[Zmiana jednostki czasu na wykresie]

W celu pobrania danych z wyznaczonego zakresu w postaci pliku CSV należy nacisnąć na przycisk "CSV". Otworzy to okienko pozwalające użytkownikowi wybranie folderu
zapisu pliku oraz zmianę jego nazwy. 

[pobieranie pliku]

[wygląd pliku]

\subsection{Widok prezentacyjny}
Aplikacja posiada również widok pozwalający na czytelne i estetyczne zaprezentowanie danych bliżej nie określonej grupie odbiorców, na przykład poprzez wyświetlenie
na telewizorze na korytarzu.
Widok ten przedstawia odczyt z 4 czujników jednocześnie. Typ odczytu zmienia się cyklicznie co 20 sekund przedstawiając kolejno odczyty temperatury, wilgotności
i dwutlenku węgla

[Widok]

W górnej części każdego z czterech elementów można odczytać podstawowe informacje, takie jak sala w której czujnik aktualnie się znajduje, typ odczytu, podstawowe
statystyki oraz jednostkę. Na każdym wykresie umieszczono horyzontalne linie odpowiadające poziomom komfortu odczytu. Tak jak w przypadku pozostałych widoków
dane wczytywane są na bieżąco - dostarczenie przez serwer nowej paczki odczytów spowoduje odświeżenie wykresów.