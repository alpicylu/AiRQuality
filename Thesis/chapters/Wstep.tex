\section{Wstęp}

Celem pracy było stworzenie systemu monitorowania jakości powietrza w wybranych salach Politechniki Wrocławskiej. Na system składają się
czujniki wraz z bramką umożliwiającą komunikację z nimi, aplikacja webowa oraz zewnętrzny serwis IQRF Cloud służący w roli pośrednika
między aplikacją a systemem czujników.
Aplikacja webowa została napisana za pomocą dostępnych frameworków i bibliotek, a czujniki, bramka i serwis zostały skonfigurowana
w sposób zapewniający poprawne działanie w tym systemie. Parametry jakości powietrza jakie system monitoruje 
to temperatura, wilgotność względna oraz stężenie dwutlenku węgla. Sama aplikacja zaś umożliwia monitorowanie danych, zarówno 
bieżących jak i historycznych.

Ze względu na zmiany klimatyczne oraz pogarszający się stan powietrza wewnętrznego, szczególnie w miesiącach zimowych, ważne jest jego 
regularne monitorowanie. Człowiek spędza znaczną część swojego życia właśnie w pomieszczeniach zamkniętych, więc powietrze w nich
będzie miało znaczny wpływ na jego zdrowie, a monitorowanie parametrów powietrza jest kluczowym elementem wpływania na ich jakość.

Zasadnicza część pracy składa się z 6 rozdziałów. Pierwszy z nich opisuje parametry jakości powietrza jakie system monitoruje, ich wpływ 
na zdrowie i komfort oraz bezpieczne wartości. Drugi opisuje ogół systemu, począwszy od czujników, przez bramkę, serwis IQRF Cloud i kończąc
na samej aplikacji. Następny omawia technologie webowe zastosowane w aplikacji i uzasadnia ich wybory. Czwarty oraz piąty opisują dogłębniej działanie
samej aplikacji, zarówno interfejsu użytkownika jak i serwera. Ostatni, zawiera obserwacje powstałe podczas tworzenia pracy.





% todo: 
% czym jest praca (aplikacja, system, czujniki, etc)
% motywacja do stwożenia pracy 
% ogolny zarys projektu
% co zawiera praca (po prostu wymien rozdziały)